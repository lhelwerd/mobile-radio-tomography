\documentclass{article}

\usepackage[a4paper,margin=2.5cm]{geometry}
\parindent=0pt
\frenchspacing

\usepackage[none]{hyphenat}
\usepackage{parskip}
\usepackage[hyphens]{url}
\usepackage{hyperref}

\hypersetup{colorlinks, citecolor=red, filecolor=red, linkcolor=black, urlcolor=blue}

\newcommand{\arch}{{\color{red} (Arch)}}

\begin{document}

\title{DroneKit, MAVProxy and ArduPilot: Working with the vehicle simulator}
\author{Tim van der Meij and Leon Helwerda}
\date{\today}

\maketitle

\section{Introduction}
The SITL (software in the loop) vehicle simulator allows us to run DroneKit 
Python scripts or other MAVProxy-based programs on a computer, i.e., without an 
actual vehicle. It uses the same ArduPilot autopilot binary. This document 
outlines the steps to successfully install the SITL simulator with all 
requirements and run Python scripts. Note that the steps here are suitable for 
any Linux distribution. Steps that have been altered for Arch Linux are 
indicated with \arch{}.

\section{Installing the simulator}
We follow instructions from the 
\href{http://ardupilot.org/dev/docs/setting-up-sitl-on-linux.html}{ArduPilot 
documentation} which uses Ubuntu Linux or Fedora as an example. Not all 
commands are exactly the same on other systems and some commands might not be 
necessary, but we will go through the bare needs first.

For all requirements, take heed of the prerequisites in the README.md file in 
this directory. Make sure Git is available and is a somewhat recent version 
first. Furthermore, make sure Python~2.7 is available.

\begin{itemize}
    \item Create a folder in your home directory to store all vehicle software 
          (referred to as \emph{vehicle\_folder} from now on). Navigate to this 
          folder: \\
          {\tt cd $\sim$/\emph{vehicle\_folder}}
    \item Download ArduPilot: \\
          {\tt \$ git clone git://github.com/diydrones/ardupilot.git}

          This downloads the latest ardupilot code into a subdirectory. In 
          order to update the code, run {\tt git pull} within that directory.

          Note that every time the code is updated, the simulator needs to be 
          recompiled. The stability of the code can change over time, so use 
          this with caution.
    \item The MAVProxy connection and other components of ArduPilot require 
          some Python packages. These might also be available in {\tt pip} by 
          using {\tt sudo pip install <package>} (or {\tt pip install --user 
          <package>} without superuser right). For some Linux systems,
          distribution packages are available instead. If you also do not have 
          these, the README in this repository may offer some other solutions.
          
          First, install {\tt pip} for Python 2: \\
          {\tt \$ su} \\
          {\tt \$ pacman -S python2-pip} \arch{} \\
          {\tt \$ exit}
    \item Install Python packages that do not require other software to be 
          installed: \\
          {\tt \$ pip2 install --user pymavlink mavproxy matplotlib lxml pexpect} \arch{}
    \item Install other required software: \\
          {\tt \$ su} \\
          {\tt \$ pacman -S ccache opencv wxpython blas lapack gcc-fortran} \arch{} \\
          {\tt \$ exit}
    \item Install SciPy (which depends on the previously installed software): \\
          {\tt \$ pip2 install --user scipy} \arch{}
    \item Edit {\tt $\sim$/.bashrc} by running {\tt vim $\sim$/.bashrc} to 
          append the following fragment: \\\\
          {\tt \# ArduPilot software} \\
          {\tt export PATH=\$PATH:\$HOME/\emph{vehicle\_folder}/ardupilot/Tools/autotest} \\
          {\tt export PATH=/usr/lib/ccache:\$PATH} \\
          {\tt export PATH=\$PATH:\$HOME/.local/bin} \\\\
          The last line is required to be able to access {\tt mavproxy.py} 
          directly.
    \item Install DroneKit for Python: \\
          {\tt \$ pip2 install --user droneapi} \arch{}
    \item Download DroneKit source code and examples: \\
          {\tt \$ git clone git://github.com/dronekit/dronekit-python.git}
\end{itemize}

\section{Running Python scripts}
Once you have written a Python script (located at \emph{script\_path}), you can 
run it using the simulator.

\begin{itemize}
    \item Start the simulator for a copter vehicle with a console and a map 
          view: \\
          {\tt \$ cd $\sim$/\emph{vehicle\_folder}/ardupilot/Tools/autotest} \\
          {\tt \$ sim\_vehicle.sh -v ArduCopter --console --map} \\
          If ArduCopter has not been compiled yet, this script will take care 
          of that. Note that there is no need for the {\tt --aircraft} 
          parameter from the original documentation.

          You can run this command in any directory, since we added the 
          directory from the first command to the {\tt \$PATH}. When you run it 
          in one of the directories for the vehicle types of ArduPilot, then 
          you do not need to pass the {\tt -v <vehicle>} parameter.
    \item Once the terminal window contains a line about saved parameters, 
          press Enter to enter command mode. Then, run the following commands 
          to configure the parameters for the copter: \\
          {\tt > param load copter\_params.parm} \\
          {\tt > param set ARMING\_CHECK 0}
\end{itemize}

\vspace{0.4cm}

The Python scripts for DroneKit~2 (to be installed with {\tt pip2 install 
--user dronekit} \arch{}) no longer use the SITL MAVProxy connection to start 
their scripts, but it is still possible to connect to the simulator. After 
starting the {\tt sim\_vehicle.sh} command, there will be output regarding a 
TCP or UDP port that the simulator itself connects to, for example {\tt 
tcp:127.0.0.1:5760}. Additionally, there is a console window that describes 
other ports that have been opened, for example {\tt 5762} or {\tt 5763}. One 
can pass a connection string line {\tt mission\_basic.py --connect 
tcp:127.0.0.1:5762} to connect to these additional ports and control the 
vehicle from there.

\end{document}

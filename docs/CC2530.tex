\documentclass{article}

\usepackage[a4paper,margin=2.5cm]{geometry}
\parindent=0pt
\frenchspacing

\usepackage[none]{hyphenat}
\usepackage{parskip}
\usepackage[hyphens]{url}
\usepackage{hyperref}
\usepackage{booktabs}

\hypersetup{colorlinks, citecolor=red, filecolor=red, linkcolor=black, urlcolor=blue}

\begin{document}

\title{Connecting the Raspberry Pi to a CC2530}
\author{Leon Helwerda}
\date{\today}

\maketitle

\section{Introduction}
The CC2530 Evaluation Module (EM) is a Texas Instruments device that support 
radiographic RF communication and measurements. An antenna mount allows the use 
of any compatible large antenna. The CC2530 has a processor that can be 
programmed to execute such measurement and packet sending operations. It is 
a system-of-chip (SoC) device containing the necessary equipment on a small 
chip. There are also similar devices, such as the CC2531 USB dongle with 
a small internal antenna.

In contrast to the CC2531, where we can use a USB hardware abstraction layer 
(HAL) library to interface with the USB host, the CC2530 needs to communicate 
via its I/O pins. Often, the CC2530 is mounted on a SoC battery board, which 
provides an LED, button, AA batteries and normal-sized male pin headers for 
I/O, debug flash as well as jumpers.

To flash a program onto the CC2530, we use a SmartRF05 Evaluation Board (EB). 
Either the battery module's flash header can be used with a female-to-female 
connector, or you can mount the disconnected CC2530 onto the EB.

The CC2530 supports UART and SPI for I/O communication. More details about all 
the connection possibilities and the actual setup are given in 
Section~\ref{sec:setup}.

In this document, we describe how we use the CC2530 and the battery board to 
connect it with the Raspberry Pi. This makes it possible to easily reproduce 
this installation.

\section{Prerequisities}

These steps only need to be done once on the Raspberry Pi to make it possible 
to use all the UART pins of the GPIO board. By default, the UART pins are used 
for a serial login interface, and the pins for hardware flow control are 
disabled by default. Because the CC2530 has a limited buffer size for receiving 
and transmitting over UART, we need to use the RTS/CTS pins for flow control.

\begin{enumerate}
  \item Ensure that the WiringPi Python library is installed for the root user, 
        so that the Python scripts can use it to set the UART pins to the 
        correct alternative modes: {\tt sudo pip2 install --user wiringpi}.
  \item Edit with {\tt sudo vim /boot/cmdline.txt} such that there are no 
        references to {\tt /dev/ttyAMA0} in that file, removing the keywords as 
        well. Basically, remove the parts {\tt console=ttyAMA0,115200} and {\tt 
        kgdboc=ttyAMA0,115200}. This disables the serial login interface of the 
        Raspberry Pi on the UART RX/TX pins, which is not necessary if the 
        network connection which we set up in the Raspberry Pi configuration 
        works.

        Warning: If the edit is done incorrectly and {\tt /boot/cmdline.txt} 
        becomes malformed, then the Raspberry Pi may be unable to boot. Make 
        a backup of this file beforehand and keep a microSD reader handy in 
        case that you are not confident to do this.
\end{enumerate}

\section{Setup}\label{sec:setup}

Perform the following steps to connect the CC2530 to the Raspberry Pi.

\begin{enumerate}
  \item Power off all the devices.
  \item Remove any batteries in the CC2530's battery module because we will 
        power it from the Raspberry Pi for the time being. The on/off switch 
        does not function when powering in this way, because the 3.3v pins are 
        directly connected to the CC2530 VDD.
  \item Compile the latest RF code using `make` in the corresponding directory. 
        Use the SmartRF05 EB to flash the program to the CC2530. One can detach 
        the evaluation module from the battery module and place it on the 
        SmartRF evaluation board, or use the debug cable to flash the program.
  \item Connect wires between the pins mentioned in Table~\ref{tab:uart} on the 
        Raspberry Pi and the CC2530. The RPi pin numbering is by physical 
        order, the CC2530 naming is by port number and label on the device. 
        Note: P0 = P10 and P1 = P11 on the battery module.

        \begin{table}[h!]
          \centering
          \begin{tabular}{rlrl}
            \toprule
            RPi physical pin & RPi name & CC2530 port & CC2530 label \\
            \midrule
            08 & TXD0 & P0.2 & RX \\
            10 & RXD0 & P0.3 & TX \\
            11 & RTS0 & P0.4 & CT \\
            17 & 3.3v DC & P0 3.3V & VDD \\
            36 & CTS0 & P0.5 & RT \\
            39 & GND & P0 GND & GND \\
            40 & reset & P1 RSn & RESET\_N \\
            \bottomrule
          \end{tabular}
          \caption{UART connection setup for the Raspberry Pi and CC2530.}
          \label{tab:uart}
        \end{table}

        Warning: When you connect output pins to each other, such as TXD0 to TX 
        or RTS to RTS, then the pins may short-circuit and malfunction when 
        power is applied to those pins. Such defects are irreversible. Ensure 
        that the pins are connected as mentioned in Table~\ref{tab:uart} to 
        prevent permanent damage.
\end{enumerate}

\section*{References}

\begin{itemize}
  \item CC2530ZDK User's Guide (Rev. B). 
    \url{http://www.ti.com/lit/pdf/swru209}
  \item CC253x/CC254x User's Guide. \url{http://www.ti.com/lit/pdf/swru191}
  \item CC SoC Battery Board User's Guide. 
    \url{http://www.ti.com/lit/pdf/swru241}
  \item CC2530 Datasheet (Rev. B). \url{http://www.ti.com/lit/pdf/swrs081}
  \item SmartRF05 Evaluation Board User's Guide. 
    \url{http://www.ti.com/lit/pdf/swru210}
\end{itemize}

\end{document}

\documentclass{article}

\usepackage[a4paper,margin=2.5cm]{geometry}
\parindent=0pt
\frenchspacing

\usepackage[none]{hyphenat}
\usepackage{parskip}
\usepackage[hyphens]{url}
\usepackage{hyperref}

\hypersetup{colorlinks, citecolor=red, filecolor=red, linkcolor=black, urlcolor=blue}

\newcommand{\arch}{{\color{red} (Arch)}}

\begin{document}

\title{Working with the vehicle simulator}
\author{Tim van der Meij}
\date{\today}

\maketitle

\section*{Introduction}
The SITL (software in the loop) vehicle simulator allows us to run DroneKit Python scripts
on a computer, i.e., without an actual vehicle. This document outlines the steps to successfully
install the SITL simulator with all requirements and run Python scripts. Note that the steps are
suitable for any Linux distribution. Steps that have been altered for Arch Linux are indicated
with \arch{}.

\section*{Installing the simulator}
\begin{itemize}
    \item Create a folder in your home directory to store all vehicle software (referred to
          as \emph{vehicle\_folder} from now on). Navigate to this folder: \\
          {\tt cd $\sim$/\emph{vehicle\_folder}}
    \item Download ArduPilot: \\
          {\tt \$ git clone git://github.com/diydrones/ardupilot.git}
    \item Install {\tt pip} for Python 2 (note that Python 3 cannot be used!): \\
          {\tt \$ su} \\
          {\tt \$ pacman -S python2-pip} \arch{} \\
          {\tt \$ exit}
    \item Install Python packages that do not require other software to be installed: \\
          {\tt \$ pip2 install --user pymavlink mavproxy matplotlib lxml pexpect} \arch{}
    \item Install other required software: \\
          {\tt \$ su} \\
          {\tt \$ pacman -S ccache opencv wxpython blas lapack gcc-fortran} \arch{} \\
          {\tt \$ exit}
    \item Install SciPy (which depends on the previously installed software): \\
          {\tt \$ pip2 install --user scipy} \arch{}
    \item Edit {\tt $\sim$/.bashrc} by running {\tt vim $\sim$/.bashrc} to append the following fragment: \\\\
          {\tt \# Drone software} \\
          {\tt export PATH=\$PATH:\$HOME/\emph{vehicle\_folder}/ardupilot/Tools/autotest} \\
          {\tt export PATH=/usr/lib/ccache:\$PATH} \\
          {\tt export PATH=\$PATH:\$HOME/.local/bin} \\\\
          The last line is required to be able to access {\tt mavproxy.py} directly.
    \item Install DroneKit for Python: \\
          {\tt \$ pip2 install --user droneapi} \arch{}
    \item Download DroneKit source code and examples: \\
          {\tt \$ git clone git://github.com/dronekit/dronekit-python.git}
\end{itemize}

\section*{Running Python scripts}
Once you have written a Python script (located at \emph{script\_path}), you can run it using the simulator.

\begin{itemize}
    \item Start the simulator for a copter vehicle with a console and a map view: \\
          {\tt \$ cd $\sim$/\emph{vehicle\_folder}/ardupilot/Tools/autotest} \\
          {\tt \$ sim\_vehicle.sh -v ArduCopter --console --map} \\
          If ArduCopter has not been compiled yet, this script will take care of that. Note that there is
          no need for the {\tt --aircraft} parameter from the original documentation.
    \item Once the terminal window contains a line about saved parameters, press Enter to enter command mode.
          Then, run the following commands to configure the parameters for the copter: \\
          {\tt > param load copter\_params.parm} \\
          {\tt > param set ARMING\_CHECK 0}
    \item Load the DroneKit API module: \\
          {\tt > module load droneapi.module.api}
    \item Finally, run the Python script: \\
          {\tt > api start \emph{script\_path}}
\end{itemize}

\vspace{0.4cm}

To simplify running Python scripts even further, run {\tt vim $\sim$/.mavinit.scr} and add the following line: \\\\
{\tt module load droneapi.module.api}

After saving the file, you can now skip the module loading step every time you use the simulator.

\end{document}

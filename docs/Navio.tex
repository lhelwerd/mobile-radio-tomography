\documentclass{article}

\usepackage[a4paper,margin=2.5cm]{geometry}
\parindent=0pt
\frenchspacing

\usepackage[none]{hyphenat}
\usepackage{parskip}
\usepackage[hyphens]{url}
\usepackage{hyperref}
\usepackage{booktabs}

\hypersetup{colorlinks, citecolor=red, filecolor=red, linkcolor=black, urlcolor=blue}

\newcommand{\navio}{Navio$\stackrel{}{+}$}

\begin{document}

\title{Setting up the \navio{} flight controller hardware}
\author{Leon Helwerda}
\date{\today}

\maketitle

\section{Introduction}
This document describes setting up the \navio{} flight controller, connecting 
it to the Raspberry Pi and making it act as the autopilot for our mission 
scripts. The setup procedure is partially based on the 
\href{http://docs.emlid.com/navio/Navio-APM/hardware-setup-navio-plus/}{setup 
instructions} by Emlid.

\section{Setup}
How to connect the \navio{} to the Raspberry Pi:

\begin{itemize}
  \item Remove the case from the Raspberry Pi. Remove the microSD card. Lower 
        the base of the case slowly, although one will need to use more force 
        than one is comfortable with. Start from the hardest long side, the one 
        *without* mini-USB and other ports. Work your way around the base until 
        it can be lifted out of it, then remove the top part so that the Pi 
        motherboard is completely free.
  \item Install the 4 gold-colored spacers using the small black screws. This 
        can be done manually without any tools, but for some of the screws we 
        might want to use a small screwdriver (hexagonal screw drive socket) 
        for secure fastening.
  \item Connect the extension header to the 40-pin GPIO ports. It should fit 
        tight.
  \item Place the \navio{} on top of the Raspberry Pi so that the extension 
        header fits through the holes and fills them completely. Fix it onto 
        the spacers using 4 more black screws.
  \item Remember to add the microSD card back in the Pi port.
\end{itemize}

Removal is basically undoing the steps in reverse order. Removing the \navio{} 
from the GPIO extender and the extenders from the Pi might be tricky when the 
pins are connected tightly, so be sure not to pull it on one side too much.

It is recommended to have a way to protect the entire circuit from air and 
other flows. You can use a 3D-printed case, or use an own, larger protective 
shield. The latter option has the advantage that we can extend it with other 
modules without problem. This will be important when we run missions in open 
air, where UV can also mess up the barometer.

\section{Power source}
For testing, we can power the Raspberry Pi and \navio{} using the micro-USB 
port. It might be possible to power it with the current battery pack via the 
power port of the \navio{}, but we need to be safe with this. During actual 
flight, the \navio{} powers the Pi using a redundant power module on this port. 
The power module needs a battery and an ESC/PDB, both using XT60 ports.

For the servo rail, we need to connect a distinct power source to power the 
servos. For testing, we can also power the servo rail, \navio{} and Pi from the 
servo rail, as long as only one power source is connected. The \navio{} shield 
protects itself and the Pi in case two distinct power sources are used. This is 
to prevent the servos from stealing too much power during actual flight. 
Certain drones and rovers have a BEC that should be connected here.

\section{GPS}
Connect the MCX GPS antenna on the golden connector on the board. Note that it 
seems very hard to remove the antenna afterward. Be sure not to destroy the 
connector when doing so.

Note that this simple antenna will easily not receive any GPS info inside 
buildings because there are no visible antennas (no GPS lock). To debug this, 
there are some useful tools in the \href{https://github.com/emlid/Navio}{Navio 
GitHub repository}. There is a simple GPS example that displays data in 
a readable way, and there is a tool to convert the GPS SPI interface to a PTY 
allowing it to be viewed in a screen or passed to a GPSd daemon. One can also 
install the uBlox configuration/inspection utility known as u-center, although 
it only works on Windows. It is possible to use a virtual machine with network 
sharing such that the Pi, host laptop and virtual machine are all in the same 
private subnet.

\section{ArduPilot}
Navio+ uses its own fork of ArduPilot for the flight controller. This work with 
the Raspberry Pi and sends motor control signals via GPIO through the \navio{}.
Follow the \href{http://docs.emlid.com/navio/Navio-APM/building-from-sources/}{ 
installation instructions} to get it working. Cross-compiling is difficult to 
set up, and building on the Raspberry Pi is quite slow but at least works. We 
have a specific fork with a branch for our own changes for the Raspberry Pi and 
\navio{}, for example to make compilation work on Arch Linux ARM or to make EKF 
local positioning work.

Follow these steps to make a build:

\begin{verbatim}
$ git clone https://github.com/lhelwerd/ardupilot.git -b navio-rpi
$ cd ardupilot
$ cd APMrover2
$ make navio
$ cp /tmp/ArduPilot.build/APMrover2.elf ~
\end{verbatim}

To speed up download, one can also pass {\tt --single-branch} to the first 
command. The compilation builds a Linux ELF binary, which we store in the home 
directory with the final command. One can also pass a build root different from 
the temporary (in-memory) directory by prefixing {\tt 
BUILDROOT=~/ardupilot/build} to the {\tt make} command, and adjusting the final 
command accordingly. This has the advantage that recompiling is faster, such as 
when manually debugging the software or updating it with {\tt git pull}. This 
is because partial object files that have not changed since then are not lost 
after a reboot. It might also be possible to use pure Linux builds (and maybe 
even cross-compiled ones), but the Navio build contains specific components for 
the Raspberry Pi and \navio{}, so it is recommended to use this build.

\section{Distance sensor}

To connect the distance sensor to the \navio{}, we use the UART port. An 
example cable connection setup is shown in Table~\ref{tab:navio_distance}.

\begin{table}[h!]
  \centering
  \begin{tabular}{llllll}
    \toprule
    Forwarded GPIO & \navio{} & DF13 port   & Male-to-male & Breadboard 
    & Distance sensor \\
    board pin      & pin name & cable color & cable color  & cable color
    & pin name        \\
    \midrule
    (2) & 5V & Red & Green & Red & POWER \\
    (8) & TX & White & Nothing & & \\
    (10) & RX & Blue & Nothing & & \\
    11 & IO17 & Yellow & Light blue & Yellow & Trigger \\
    12 & IO18 & Green & Purple & Green & Echo \\
    (6) & GND & Black & Gray & Blue & GND \\
    \bottomrule
  \end{tabular}
  \caption{Example connection setup for distance sensor on \navio{}}
  \label{tab:navio_distance}
\end{table}

After setting this us, start the distance sensor test script with
{\tt sudo python2 distance\_sensor\_physical.py --trigger-pin 11 --echo-pin 
12}.

After testing that it works, change the settings to use these pins by default 
for the mission scripts.

\section{Startup}
We have to start up the mission script or MAVLink ground station {\it before} 
we start the autopilot. This is because the flight controller will otherwise 
spew read errors if it cannot send its messages somewhere else. The ground 
station can run on the Pi or on a connected laptop (during development). The 
ground station connection  and can be realized using mavproxy or a full-fledged 
Mission Planner. For the former, ensure that the mavproxy Python library is 
installed and the path to {\tt mavproxy.py}, such as {\tt /root/.local/bin/}, 
is in the {\tt \$PATH} environment variable of the user running that command 
(the root user in the case of the Pi, as shown below).

Take heed of the following notes while setting up the autopilot:

\begin{itemize}
  \item For all commands using {\tt su} below, enter the root password (root), 
        while for {\tt sudo}, enter the sudoer password (alarm).
  \item Here, we prefix normal commands with {\tt \$}, commands run as su with 
        {\tt \#} and key presses with {\tt >}.
  \item Replace the IP addresses {\tt 10.42.0.10} below if we are using another 
        Pi.
\end{itemize}

The following commands and key presses set up for the flight controller and 
ground station on one and the same Pi:

\begin{verbatim}
$ cd ~
$ screen
> CTRL-a c (to open another window)
$ su
# export PATH="$PATH:/root/.local/bin"
# mavproxy.py --master 10.42.0.10:14550
> CTRL-a p (to toggle to the previous window)
$ sudo ./APMrover2.elf -A udp:10.42.0.10:14550
\end{verbatim}

Toggle back to the next window with {\tt > CTRL-a n} and one should see startup 
messages. To replace the MAVLink with out mission script, run {\tt python 
mission\_basic.py --connect 10.42.0.10:14550} instead. Make sure that all the 
settings are correct.

If running the ground station on the computer, follow these steps {\it first}:
\begin{itemize}
  \item Make sure that the ModemManager package is not installed, remove it if 
        possible.
  \item If the computer has a firewall, either temporarily disable it or allow 
        connections on port 14550 for TCP {\it and} UDP.
  \item Start the ground station. Note the use of the IP address of the master. 
        Use {\tt mavproxy.py --master 10.42.0.1:14550 --console} to start it on 
        the laptop. One can also start the APM Planner 2.
\end{itemize}

Then on the Pi, run the command {\tt sudo ./APMrover2.elf -A 
udp:10.42.0.1:14550}. Note the difference in IP addresses from the commands in 
the earlier example.

\end{document}

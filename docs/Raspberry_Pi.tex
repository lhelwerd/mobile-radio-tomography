\documentclass{article}

\usepackage[a4paper,margin=2.5cm]{geometry}
\parindent=0pt
\frenchspacing

\usepackage[none]{hyphenat}
\usepackage{parskip}
\usepackage[hyphens]{url}
\usepackage{hyperref}

\hypersetup{colorlinks, citecolor=red, filecolor=red, linkcolor=black, urlcolor=blue}

\newcommand{\arch}{{\color{red} (Arch)}}
\newcommand{\navio}{Navio$\stackrel{}{+}$}

\begin{document}

\title{Setting up a Raspberry Pi with Arch Linux ARM}
\author{Tim van der Meij}
\date{\today}

\maketitle

\section{Introduction}
This document describes setting up a Raspberry Pi device with the Arch Linux ARM
operating system. We specifically use Arch Linux ARM as it is both convenient 
and always has the most recent packages.

\section{Requirements}
This document assumes that a laptop with (standard) Arch Linux and some 
graphical interface is used to connect to the Raspberry Pi. Other distributions 
also work, but the required packages for the tools used (marked with \arch{}) 
as well as the appearance of the network manager settings may differ.

The following hardware is required for setting up the Raspberry Pi:

\begin{itemize}
    \item An Ethernet cable.
    \item A micro USB charger that is compatible with the Raspberry Pi.
    \item A micro SD card that is at least 2~GB in size. The basic installation 
          of Arch Linux ARM takes very little space and at the end around
          1.5~GB is in use.
    \item A card reader compatible with micro SD cards that can be connected to 
          the laptop via USB\@. Alternatively, a micro SD to regular SD adapter 
          can be used if the laptop has a built-in SD card reader.
\end{itemize}

\section{Operating system configuration steps}
Depending on the configuration of your laptop you might need root privileges to 
execute the steps below.

\begin{itemize}
    \item Connect the micro SD card to the laptop. Use {\tt lsblk} to find out 
          the device name, now called \emph{device\_name} (example: {\tt 
          /dev/sdc}).
    \item Start {\tt cfdisk \emph{device\_name}} and remove any existing 
          partitions. Then create two new partitions:
          \begin{itemize}
              \item Create a W95 FAT32 (LBA) partition (using the type change 
                    functionality) of 100 MiB.
              \item Create a Linux partition using the rest of the available 
                    space.
          \end{itemize}
          Then write the partition table to disk.
    \item Make sure that the {\tt dosfstools} \arch{} package is installed.  
          Run {\tt pacman -S dosfstools} \arch{} if you do not have it yet to 
          make use of {\tt mkfs.vfat}.
    \item Create a dedicated folder for the Raspberry Pi configuration on the 
          hard disk of the laptop, now called \emph{pi\_dir}.
          Then create and mount a FAT file system for the boot partition:

          {\tt mkfs.vfat \emph{device\_name}1} \\
          {\tt cd \emph{pi\_dir}} \\
          {\tt mkdir boot} \\
          {\tt mount \emph{device\_name}1 boot}
    \item Create and mount an ext4 file system for the storage partition:

          {\tt mkfs.ext4 \emph{device\_name}2} \\
          {\tt mkdir root} \\
          {\tt mount \emph{device\_name}2 root}
    \item Make sure {\tt wget} is installed (otherwise use {\tt pacman -S wget} 
          \arch{}). Download the Arch Linux ARM image:

          {\tt wget 
          http://os.archlinuxarm.org/os/ArchLinuxARM-rpi-2-latest.tar.gz}
    \item Extract the archive with {\tt bsdtar -xpf 
          ArchLinuxARM-rpi-2-latest.tar.gz -C root}. After that run {\tt sync} 
          to make sure that the files are written to the micro SD card.
    \item Move boot files to the boot partition with {\tt mv root/boot/* boot} 
          and run {\tt sync} again.
    \item Edit {\tt root/etc/systemd/network/eth0.network} so that its contents 
          are the following lines:

          {\tt [Match]} \\
          {\tt Name=eth0}
          \vspace{\baselineskip} \\
          {\tt [Network]} \\
          {\tt Address=10.42.0.10/24} \\
          {\tt Gateway=10.42.0.1} \\
          {\tt DNS=10.42.0.1}

          This ensures that the private IP address is always the same. The 
          laptop provides a DNS service via {\tt dnsmasq} that always works.  

          This example is for the first Raspberry Pi. For the second device, 
          change {\tt 10.42.0.10} to {\tt 10.42.0.11}, and so on. If at any 
          moment after these changes you cannot SSH into the device, run {\tt 
          nmap -sP 10.42.0.10/24} on the laptop to determine which IP the 
          device assigned itself.
    \item Create a file {\tt root/etc/modprobe.d/wireless.conf} to add the 
          following line:

          {\tt blacklist cfg80211}

          This disables the wireless kernel module, which we do not use since 
          the Raspberry Pi 2 has no built-in wireless support and the XBees do 
          not need it. It causes a lot of spam in the kernel logs, so it is 
          better to blacklist it.
    \item Unmount the partitions with {\tt umount boot root}. Finally, run {\tt 
          rm -rf~../\emph{pi\_dir}} to remove the temporary directory.
    \item Insert the micro SD card into the Raspberry Pi device. Connect an 
          Ethernet cable from the Raspberry Pi to the laptop and then power the 
          Raspberry Pi using a micro USB charger (in that order).
    \item On your laptop, create a new wired connection in the network manager.  
          Set the type to ``Shared with other computers'' in the IPv4 tab. Make 
          sure that {\tt dnsmasq} is installed on the laptop (otherwise use 
          {\tt pacman -S dnsmasq} \arch{}). Then select this connection in the 
          network manager to use it. Wait until the connection is established, 
          then SSH into the Raspberry Pi using {\tt ssh alarm@10.42.0.*}, 
          replacing {\tt *} with the last part of the IP address set earlier 
          on. You should now be able to run {\tt ping -c 3 www.google.com} 
          successfully.
    \item From now on we work on the Raspberry Pi device over SSH\@. Use {\tt 
          su} to become root user (the password is {\tt root}). Run {\tt 
          timedatectl set-ntp true} to update the system clock. Run {\tt ln -s 
          /usr/share/zoneinfo/Europe/Amsterdam /etc/localtime} to set the time 
          zone to UTC+2, for example.
    \item Edit the mirror list for the package manager. Run {\tt nano 
          /etc/pacman.d/mirrorlist} and remove the comments in front of the 
          lines in the section of the Netherlands, for example.
    \item Give the Raspberry Pi device a unique and identifiable hostname: run 
          {\tt echo raspberry-pi-1 > /etc/hostname} to name the device 
          ``raspberry-pi-1'' for the first Raspberry Pi, ``raspberry-pi-2'' for 
          the second Raspberry Pi, and so on.
    \item Run {\tt nano /etc/hosts} and append a tab character and the hostname 
          at the end of each line that already mentions ``localhost''.
    \item Run {\tt pacman -Syyu} to update the system.
    \item Run {\tt pacman -S git} for the Git version control system.
    \item Run {\tt pacman -S make} to be able to use Makefiles.
    \item Run {\tt pacman -S screen} to use {\tt screen} for running Python
          scripts.
    \item Run {\tt pacman -S python2 python2-pip} to install Python 2 with the 
          package manager.
    \item Run {\tt pacman -S vim} to install Vim.
    \item Run {\tt pacman -S gcc} as {\tt gcc} will be used for compiling 
          C-based Python packages such as {\tt RPi.GPIO}.
    \item Run {\tt pacman -S sudo}. Edit {\tt /etc/sudoers} by copying the line 
          containing ``root'' under ``User privilege specification'' and 
          changing ``root'' to ``alarm''.
    \item Reboot the device using {\tt reboot}.
    \item Clone the repository: \\ {\tt git clone 
          https://github.com/timvandermeij/mobile-radio-tomography}.
    \item Copy the configuration files for {\tt vim} and {\tt top}:
        
          {\tt cp mobile-radio-tomography/docs/raspberry-pi/toprc 
          $\sim$/.toprc} \\
          {\tt cp mobile-radio-tomography/docs/raspberry-pi/vimrc 
          $\sim$/.vimrc}
    \item Install the prerequisites mentioned in the README file in the 
          repository. It is important to install these as the root user. Note 
          that GUI-related packages (such as PyQt4, PyQtGraph, matplotlib, all 
          packages listed under ``Environment simulation'' and everything 
          related to ArduPilot) do not need to be installed on the Raspberry Pi 
          devices. Furthermore we recommend to install some large packages 
          using the system's package manager as that saves quite some 
          compilation time. Use {\tt pacman -S python2-numpy}, {\tt pacman -S 
          python2-scipy} and {\tt pacman -S python2-lxml} to install those 
          packages.
\end{itemize}

Make sure to update the Raspberry Pi frequently using {\tt sudo pacman -Syu}. 
This may cause these instructions to become outdated or insufficient, so check 
if everything still works after system upgrades.

\section{Distance sensor configuration steps}
\begin{itemize}
    \item Connect the HC-SR04 distance sensor on the breadboard as in
          \url{http://www.bytecreation.com/blog/2013/10/13/raspberry-pi-ultrasonic-sensor-hc-sr04}
          (left side of the image). You can use GPIO pin 2 for UUC, pin 6 for
          ground, pin 11 for echo and pin 13 for trigger. Take a look at
          \url{http://www.element14.com/community/servlet/JiveServlet/previewBody/73950-102-4-309126/GPIO_Pi2.png}
          for an overview of the GPIO pins of the Raspberry Pi 2 model B and
          \url{http://pi4j.com/images/j8header-photo.png} for how the pins are
          numbered. For connecting the distance sensor to the \navio{} refer to 
          the separate documentation related to this subject.
    \item Connect to the Raspberry Pi via SSH using {\tt ssh alarm@10.42.0.*}.
    \item Run the distance sensor test script with {\tt sudo python2 
          distance\_sensor\_physical.py}. Distance measurements should be 
          displayed continuously.
\end{itemize}

\section{Autostart configuration steps}
After booting the Raspberry Pi we immediately want to run the mission script 
instead of starting the script manually using an Ethernet cable every time. To 
maintain control over the process we wish to run the process in a detached 
screen. To achieve all this, the following actions have to be performed:

\begin{itemize}
    \item Make sure that the repository is cloned in
          {\tt /home/alarm/mobile-radio-tomography}.
    \item Copy the file {\tt docs/raspberry-pi/mobile-radio-tomography.service} 
          to {\tt /etc/systemd/system}.
    \item Run {\tt sudo systemctl enable mobile-radio-tomography.service}.
    \item Reboot the device. After rebooting, the mission script should be 
          running. Depending on settings, one can verify this by checking if 
          the LEDs on the XBee sensor are blinking, if the serial/direct 
          interface to the robot vehicle is active or if the autopilot is 
          starting. One can also log in and check that running {\tt sudo screen 
          -r sensor} opens the detached screen or check whether log files 
          appear in the {\tt logs} directory.
\end{itemize}

\section{References}
\begin{itemize}
    \item More information about Arch Linux ARM for Raspberry Pi can be found 
          at \\
          \url{http://archlinuxarm.org/platforms/armv7/broadcom/raspberry-pi-2}.
    \item Refer to the installation guide at 
          \url{https://wiki.archlinux.org/index.php/Beginners'_guide} for 
          generic steps for installing (regular) Arch Linux. Many steps do not 
          apply since we take care of partitioning and installing ourselves.
\end{itemize}

\end{document}

\documentclass{article}

\usepackage[a4paper,margin=2.5cm]{geometry}
\parindent=0pt
\frenchspacing

\usepackage[none]{hyphenat}
\usepackage{parskip}
\usepackage[hyphens]{url}
\usepackage{hyperref}

\hypersetup{colorlinks, citecolor=red, filecolor=red, linkcolor=black, urlcolor=blue}

\newcommand{\arch}{{\color{red} (Arch)}}

\begin{document}

\title{Setting up a Raspberry Pi with Arch Linux ARM}
\author{Tim van der Meij}
\date{\today}

\maketitle

\section{Introduction}
This document describes setting up a Raspberry Pi device with the Arch Linux ARM
operating system. We specifically use Arch Linux ARM as it is both convenient 
and always has the most recent packages.

\section{Requirements}
This document assumes that a laptop with (standard) Arch Linux and some 
graphical interface is used to connect to the Raspberry Pi. Other distributions 
also work, but the required packages for the tools used (marked with \arch{}), 
as well as the appearance of the network manager settings may differ.

The following cables and other hardware are required for setting up the 
Raspberry Pi:

\begin{itemize}
    \item An Ethernet cable.
    \item A micro USB charger that is compatible with the Raspberry Pi.
    \item A micro SD card that is at least 2~GB in size. The basic installation 
          of Arch Linux ARM takes very little space and at the end around
          1.5~GB is in use.
    \item A card reader compatible with micro SD cards that can be connected to 
          the laptop via USB\@. Alternatively, a micro SD-to-SD adapter can be 
          used if the laptop has a built-in SD card reader.
\end{itemize}

Additionally, it is useful to keep an HDMI cable, a USB keyboard and a USB 
mouse handy in case the setup goes wrong and makes it impossible to log in to 
the Raspberry Pi via SSH\@. One could also wipe the micro SD card and start
over.

\section{Operating system configuration steps}
\begin{itemize}
    \item Connect the micro SD card to the laptop. Use {\tt lsblk} to find out 
          the device name, now called \emph{device\_name} (example: {\tt 
          /dev/sdc}).
    \item Start {\tt cfdisk \emph{device\_name}} and remove any existing 
          partitions. Then create two new partitions:
          \begin{itemize}
              \item Create a W95 FAT32 (LBA) partition (using the type change 
                    functionality) of 100 MiB.
              \item Create a Linux partition using the rest of the available 
                    space.
          \end{itemize}
          Then write the partition table to disk.
    \item Run {\tt pacman -S dosfstools} \arch{} to be able to make use of {\tt 
          mkfs.vfat}.
    \item Create a dedicated folder for the Raspberry Pi configuration on the 
          hard disk of the laptop, now called \emph{pi\_dir}.
          Then create and mount a FAT file system for the boot partition: \\\\
          {\tt mkfs.vfat \emph{device\_name}1} \\
          {\tt cd \emph{pi\_dir}} \\
          {\tt mkdir boot} \\
          {\tt mount \emph{device\_name}1 boot}
    \item Create and mount the ext4 file system for the storage partition: \\\\
          {\tt mkfs.ext4 \emph{device\_name}2} \\
          {\tt mkdir root} \\
          {\tt mount \emph{device\_name}2 root}
    \item Make sure {\tt wget} is installed (otherwise use {\tt pacman -S wget} 
          \arch{}). Download the Arch Linux ARM image: \\\\
          {\tt wget 
          http://archlinuxarm.org/os/ArchLinuxARM-rpi-2-latest.tar.gz}
    \item Extract the archive with {\tt bsdtar -xpf 
          ArchLinuxARM-rpi-2-latest.tar.gz -C root}.
          % Tim, didn't you need to install some package to use bsdtar?
          After that run {\tt sync} to make sure that the files are written to 
          the micro SD card.
    \item Move boot files to the boot partition with {\tt mv root/boot/* boot} 
          and then run {\tt sync} again to be safe.
    \item Set up a static IP for the Raspberry Pi: \\\\
          {\tt cd boot} \\
          {\tt cp cmdline.txt cmdline.txt.bak} \\
          {\tt vim cmdline.txt} \\\\
          In that file, add a space and {\tt ip=169.254.0.2} at the end of the 
          line. Do not create a new line for that. Save the file and {\tt sync} 
          again.
    \item Place the file {\tt raspberry-pi/network} in {\tt 
          root/etc/conf.d/network} and the file {\tt 
          raspberry-pi/\\network.service} in {\tt 
          root/etc/systemd/system/network.service}. Create the directories 
          where necessary. Increment the last part of the IP address in the 
          {\tt network} file. The first Raspberry Pi should have IP address 
          {\tt 10.42.0.10} and the second Raspberry Pi has IP address {\tt 
          10.42.0.11}, and so on.
    \item Unmount the mounts using {\tt umount boot root}. Finally, run {\tt 
          rm -rf ../\emph{pi\_dir}} to remove the temporary directory.
    \item Insert the micro SD card into the Raspberry Pi device. Connect an 
          ethernet cable from the Raspberry Pi to a laptop and then power the 
          Raspberry Pi using a micro USB charger (in that order, as the 
          Raspberry Pi will boot automatically).
    \item On the laptop make sure that {\tt net-tools} is installed (otherwise 
          use {\tt pacman -S net-tools} \arch{}) for {\tt ifconfig}. You need 
          to set up a ``Link-local'' connection on your laptop. Depending on 
          your distribution you can find this in the network settings. Create 
          a new wired connection, go to the IPv4 tab and set the connection 
          type to ``Link-local''. After a while you should see an IP address in 
          {\tt ifconfig} when you activate the connection. This is when the 
          connection is established.
    \item SSH into the Raspberry Pi: {\tt ssh alarm@169.254.0.2}. Use the 
          password ``alarm''.
    \item Run {\tt su} to become the root superuser. Use the password ``root''.
    \item Edit {\tt /etc/ssh/sshd\_config} and set {\tt PermitRootLogin} to the 
          value {\tt yes} to allow root login from SSH.
    \item Run {\tt systemctl disable dhcpcd@eth0.service} and {\tt systemctl 
          enable network.service}.
    \item Edit {\tt /boot/cmdline.txt} to remove the recently added IP part. 
          After that, reboot the Raspberry Pi device.
    \item On your laptop, edit the old link-local connection. Change the type 
          to ``Shared with other computers'' and check the checkbox at the 
          bottom. Make sure that {\tt dnsmasq} is installed on the laptop 
          (otherwise use {\tt pacman -S dnsmasq} \arch{}). Then select this 
          connection in the network manager to connect to it. Wait until the 
          connection is established, then SSH to the Raspberry Pi again using 
          {\tt ssh root@10.42.0.*}. Replace {\tt *} with the last part of the 
          IP address set earlier on. You should now be able to run {\tt ping -c 
          3 www.google.com} successfully.
    \item Now we can configure Arch Linux ARM after having full SSH and 
          internet access on the Raspberry Pi. Start by setting the time: run 
          {\tt timedatectl set-ntp true} and {\tt timedatectl set-timezone 
          Europe/Amsterdam}.
    \item Edit the mirror list for the package manager: run {\tt nano 
          /etc/pacman.d/mirrorlist} and uncomment the line in the section of 
          the Netherlands.
    \item Set the hostname: run {\tt echo raspberry-pi-1 > /etc/hostname} to 
          name the device ``raspberry-pi-1'' for the first Raspberry Pi, 
          ``raspberry-pi-2'' for the second Raspberry Pi, and so on.
    \item Run {\tt nano /etc/hosts} and append a tab character and the hostname 
          at the end of each line that already mentions ``localhost''.
    \item Run {\tt pacman -Syyu} to update the system.
    \item Run {\tt pacman -S vim} to install Vim. Python 2 will be installed 
          along with this (otherwise install it manually).
    \item Run {\tt pacman -S python2-pip} to install the Python 2 package 
          manager.
    \item Run {\tt pacman -S git} for the Git version control system.
    \item Run {\tt pacman -S make} to be able to use Makefiles.
    \item Update the Raspberry Pi frequently using {\tt pacman -Syu}.
\end{itemize}

\section{Distance sensor configuration steps}
\begin{itemize}
    \item Connect the HC-SR04 distance sensor on the breadboard as in
          \url{http://www.bytecreation.com/blog/2013/10/13/raspberry-pi-ultrasonic-sensor-hc-sr04}
          (left side of the image). You can use pin 2 for UUC, pin 6 for
          ground, pin 11 for echo and pin 13 for trigger. Take a look at
          \url{http://www.element14.com/community/servlet/JiveServlet/previewBody/73950-102-4-309126/GPIO_Pi2.png}
          for an overview of the GPIO pins for the Raspberry Pi model 2B and
          \url{http://pi4j.com/images/j8header-photo.png} for how the pins are
          numbered.
    \item Connect to the device via SSH using {\tt ssh alarm@10.42.0.*}, become
          root user using {\tt su} and run {\tt pacman -S gcc sudo}.
    \item Run {\tt vim /etc/sudoers}, copy the root user line under ``User
          privilege specification'', change ``root'' to ``alarm'' and
          force-save.
    \item Run {\tt pip2 install rpi.gpio} (still as root).
    \item Run {\tt su alarm} and place the file {\tt raspberry-pi/vimrc} in
          {\tt $\sim$/.vimrc}.
    \item Copy the file {\tt distance\_sensor.py} to the device: \\\\
          {\tt scp distance\_sensor.py alarm@10.42.0.*:$\sim$/distance\_sensor.py}.
    \item Run the distance sensor script with {\tt sudo python2 distance\_sensor.py}
          and distance measurements should be displayed continuously.
\end{itemize}

\section{References}
\begin{itemize}
    \item Arch Linux ARM can be obtained from \url{http://archlinuxarm.org/platforms/armv7/broadcom/raspberry-pi-2}.
    \item See \url{http://bvanderveen.com/a/rpi-booted-static-ip-ssh} and 
          \url{http://l2ork.music.vt.edu/main/?page_id=2288} about setting up 
          the connection to the Raspberry Pi and making it connect to the 
          internet via network sharing.
    \item See the installation guide at 
          \url{https://wiki.archlinux.org/index.php/Beginners'_guide} for 
          generic steps for installing (normal) Arch Linux. Many steps do not 
          apply since we take care of partitioning and installing ourselves.
\end{itemize}

\end{document}

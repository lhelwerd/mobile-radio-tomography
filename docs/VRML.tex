\documentclass{article}

\usepackage[a4paper,margin=2.5cm]{geometry}
\parindent=0pt
\frenchspacing

\usepackage[none]{hyphenat}
\usepackage{parskip}
\usepackage{hyperref}

\hypersetup{colorlinks, citecolor=red, filecolor=red, linkcolor=black, urlcolor=blue}

\begin{document}

\title{VRML files}
\author{Leon Helwerda}
\date{\today}

\maketitle

\section{Usage notes}
There are VRML files at 
\url{http://castle-engine.sourceforge.net/demo_models.php} that can be used to 
simulate environments. Interesting ones are {\tt vrml\_2/castle.wrl}, {\tt 
shadow\_maps/trees\_river.wrl} and {\tt 
sensors\_environmental/deranged\_house.wrl} (indoor).

Certain files use nonstandard/unsupported/new features that cause the VRML 
parser to give up. The solution is to remove the entire node that the parser 
mentions as problematic, since it is a feature that we do not use.

One can export scene files in Blender using an addon that is provided with the 
download, but disabled by default. Go to ``File $\to$ User Preferences $\to$ 
Add-ons $\to$ Categories: Import-Export'' and scroll down to the VRML2 addon.  
Check the box next to it and save the settings. One can import PLY, OBJ, 
Autodesk 3D Studio (somewhat open/standard formats) or X3D files (the followup 
of VRML) and then export them to VRML in this way. Note that this might cause 
certain information to be lost, but we only use the bare polygon coordinate 
information, so that is not a problem. It does seem that the orientation of the 
scene is sometimes altered.

\end{document}

\documentclass{article}

\usepackage[a4paper,margin=2.5cm]{geometry}
\parindent=0pt
\frenchspacing

\usepackage[none]{hyphenat}
\usepackage{parskip}
\usepackage[hyphens]{url}
\usepackage{hyperref}

\hypersetup{colorlinks, citecolor=red, filecolor=red, linkcolor=black, urlcolor=blue}

\begin{document}

\title{Configuring XBee sensors}
\author{Tim van der Meij}
\date{\today}

\maketitle

\section{Introduction}
This document describes the configuration steps required to setup an XBee sensor
for usage in the sensor network. We use XBee Pro 60 mW series 2 wire antenna
sensors, each equipped with an XBee Explorer USB dongle.

\section{Configuration}
Perform the following steps to configure a single sensor. Repeat most of these
steps for each sensor in the network.

\begin{itemize}
    \item Edit {\tt /etc/sudoers} to add your username such that {\tt sudo} can
          be used.
    \item Install XCTU for Linux from \url{http://www.digi.com/xctu} and run it
          using {\tt sudo~./launcher}. {\tt sudo} is required because we need
          permission to use the USB ports if we are not in the ``dialout'' user
          group.
    \item Plug in the XBee sensor, locate it using XCTU and update the firmware:
          \begin{itemize}
              \item Product family: XBP24BZ7
              \item Function set: ZigBee End Device API
              \item Firmware version: Newest
          \end{itemize}
    \item In XCTU change the SM setting to Pin Hibernate (to avoid issues with
          the XBee sensor intermittently not responding to calls) and change the
          BD (baud rate) setting to 57600.
\end{itemize}

It is important that exactly one sensor (namely the ground sensor) in the network
is the coordinator (with function set ZigBee Coordinator API) as that sensor
takes care of initializing the network. Without a coordinator the network will
not function correctly!

When all sensors are configured, we can update the other settings such as PAN ID
and node ID\@. The settings mentioned above only have to be updated once, whereas
the configurator takes care of settings that might change more frequently.

\begin{itemize}
    \item Run {\tt pip2 install --user pyserial xbee} both as regular user and
          as root.
    \item Run {\tt sudo python2 xbee\_configurator.py} to start the configurator.
          This tool will set the right settings for each connected XBee sensor.
\end{itemize}

\end{document}
